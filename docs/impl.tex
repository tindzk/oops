\documentclass{guide}

\usepackage{url}

\title{OOPS Compiler Implementation Details}
\date{\today}

%\lstset{
%  numbers=left,
%  frame=single,
%  breaklines=true
%}

% A statement.
\newcommand{\st}[1]{\texttt{#1}}
% A constant.
\newcommand{\const}[1]{\texttt{#1}}
\newcommand{\type}[1]{\texttt{#1}}
\newcommand{\keyword}[1]{\texttt{#1}}
\newcommand{\oper}[1]{\texttt{#1}}

\begin{document}
\maketitle
\tableofcontents

\chapter{Introduction}
OOPS is a simple object-oriented programming language for teaching purposes. In
this document, I am outlining my implementation of a compiler for the OOPS
language. The compiler produces assembly code to be run by a virtual machine.

The project was developped within the scope of the lecture
``Übersetzerpraktikum'' held at the University of Bremen, Germany during the
summer semester 2013. Even though a preliminary compiler, written in Java, was
provided, I chose to rewrite it in Scala, resulting in a number of code
reductions due to the expressiveness and the functional programming features of
this language.

If the following I refer to the compiler by \textit{oopsc} and the virtual
machine as \textit{oopsvm}.

\chapter{Overview}
	\section{Language features}
	\begin{itemize}
		\item Simple and consistent syntax.
		\item Limited language features; reduced to essential functionality.
		\item Only two system calls for I/O; for processing standard input/output.
		\item Exception support
			\begin{itemize}
				\item Exceptions can be caught or propagated.
				\item Exception is thrown upon NULL pointer dereferencing.
				\item Exception is thrown upon ``division by zero''.
				\item Exception codes are Integer literals (or characters).
				\item A default exception handler is defined.
			\end{itemize}
		\item Object-oriented.
			\begin{itemize}
				\item Methods have return values, parameters.
				\item Classes have attributes.
				\item Single inheritance with cycle detection.
				\item Method overloading.
				\item Late state binding: Virtual method calls and virtual method tables.
				\item Access levels: public, private, protected.
				\item Run-time type casts, type checking.
				\item Primitive data types are exposed as class objects.
				\item Automatic boxing/unboxing faciliates access to primitive data types.
			\end{itemize}
		\item Primitive data types: Void, NULL, Integer, Boolean, String.
		\item Class data types: Object, Boolean, Integer.
		\item Code optimisations: Evaluation of constant expressions.
		\item Control structures: if-conditions, while-loops.
		\item Statically typed.
		\item Entry point is always \texttt{Main.main()}.
		\item No garbage collection.
		\item VM is register-based (in total 7), stack is heap size is static.
	\end{itemize}

\chapter{Installation}
	\section{Compilation from the sources}
		\subsection{Prerequisites}
		For compiling oopsc manually, the following libraries and programs have to be
		installed on the host system:

		\begin{itemize}
			\item Scala 2.10
			\item ANTLR4
			\item Python 3
			\item Java 1.7
		\end{itemize}

		\subsection{Compilation}
		To compile oopsc manually, we need to generate the grammar first:
\begin{verbatim}
./grm.sh
\end{verbatim}

		Then, we have to copy the Scala runtime libraries \texttt{scala-library.jar}
		and \texttt{scala-reflect.jar} to \texttt{libs/}. These can be usually found
		in the directory \texttt{/usr/share/scala/lib/}.

		Finally, the script \texttt{build.py} is used to compile the whole project,
		producing \texttt{oopsc.jar} and \texttt{oopsvm.jar}.

	\section{Execution}
		\subsection{Prerequisites}
		For running oopsc or oopsvm on the target system the only required
		prerequisites are the compiled JAR files  \texttt{oopsc.jar} and
		\texttt{oopsvm.jar} as well as Java 1.7. The files are self-contained and
		include already all dependencies.

\chapter{Getting started}
	The available options of the compiler can be optained as follows:

\begin{verbatim}
$ java -jar oopsc.jar --help
oopsc 0.1 (c) 2013 Tim Nieradzik
Usage: java -jar oopsc.jar [OPTION]... [input] [<output>]
oopsc is an OOPS compiler.

Options:

  -a, --ast                 print AST after contextual analysis
  -c, --code                enable code generation (default)
      --nocode              disable code generation
  -d, --debug               enable debug mode
      --heap-size  <arg>    heap size (default = 100)
  -o, --optim               enable optimisations
      --stack-size  <arg>   stack size (default = 100)
  -s, --symbols             show symbols from the syntax analysis
  -h, --help                print help
      --version             Show version of this program

 trailing arguments:
  input (required)        input file
  output (not required)   output file (default: stdout)
\end{verbatim}

\chapter{Implementation}
	This chapter is meant to provide pointers as to how and where the particular
	tasks were implemented. Note that the code is sufficiently annotated with
	comments describing the functioning or the reasoning behind the chosen approach
	in more detail.

	\section{Structural changes}
	The original compiler used a self-written parser for the syntax analysis. This
	was replaced by a ANTLR4 grammar which is transformed directly into Java code.
	This change allowed a faster implementation of further language extensions and
	made the code parsing less error-prone while providing effortless error
	reporting.

	Minor changes to the interfaces of the virtual machine had to be made in order
	to allow seamless testing using JUnit. The instruction set remains compatible,
	though.

	Previously, the syntax tree was rewritten during the contextual analysis. This
	lead to code that was difficult to grasp and to debug. But the previous
	approach also massively created unnecessary objects. This rewriting was used
	for inserting \texttt{SELF} into the syntax tree when referring to a local method or
	attribute. This was made redundant by storing a context variable in
	\texttt{EvaluateExpression} which would by default point to the \texttt{SELF} variable
	of the current method (given that the expression that is to be evaluated is a
	method or an attribute). Another case where the syntax tree was rewritten is
	boxing. The logic was merged and moved into \texttt{Expression}.

	Another invasive change was the introduction of a more modular symbol table. It
	was designed to resemble the approach described in
	\cite[p.~182ff]{Parr:2009:LIP:1823613}. This change resulted in a separation of
	the contextual analysis into three passes: a definition pass, a referential
	pass and an (optional) optimisation pass.

	Finally, a read-only data segment (\textit{rodata}) in the resulting assembly code is
	created. This made the implementation of strings (as opposed to characters)
	trivial. oopsc implements rodata in a similar way as in the binary format ELF
	with the only difference that the first element of each entry contains its size
	whereas in ELF the rodata elements are usually NUL-terminated strings.

	\section{Approaches for the development}
	The entire project was developped using the distributed version control system
	Git. This increased the pace of the development. For instance, occasionally
	introduced bugs could be found relatively easy by bisecting the responsible
	commit.

	The chosen development approach was test-driven development (TDD) using JUnit4,
	also contributing to reducing the commit sizes as well as obtaining a better
	overview as to what is required to implement a certain feature. At the same
	time this ensured the correct functioning of all language features and the
	interoperability thereof. Regressions could be detected more quickly, reducing
	the development time.

	\section{TRUE, FALSE}
	\texttt{TRUE} and \texttt{FALSE} being literals, they appear in the grammar
	(\texttt{Grammar.g4}) under the rule \texttt{literal}. By analogy, an
	\texttt{BooleanLiteralExpression} object is instantiated during the
	syntax analysis phase, see also \texttt{SyntaxAnalysis.getLiteral()}.

	\section{ELSE, ELSEIF}
	Similarly to the implementation of \texttt{TRUE} and \texttt{FALSE} the grammar
	was extended accordingly. This happens in the rule \texttt{statement}:

\begin{verbatim}
'IF' expression 'THEN' statements
('ELSEIF' expression 'THEN' statements)*
('ELSE' statements)?
'END IF'
\end{verbatim}

    In \texttt{SyntaxAnalysis.getIfStatement()} the \texttt{ELSEIF} and
    \texttt{ELSE} blocks are being taken into account.

    \texttt{IfStatement} stores all \texttt{IF} and \texttt{ELSEIF} branches in
    a list of tuples containing the expression as well as the statements. The
    final \texttt{ELSE} branch is stored separately.

    In \texttt{generateCode()} a label is generated for each branch. When the
    current branch expression does not evaluate to \texttt{true}, a jump to the
    next expression check is performed. This is repeated as long as no
    \texttt{ELSEIF} branches is left. If the last expression also evaluated to
    \texttt{false}, then the \texttt{ELSE} block is executed.

	\section{Boolean arithmetic (AND, OR, NOT)}
	In ANTLR4 the order of subrules defines implicitly their precedence. The
	implementation in the grammar for \texttt{AND}, \texttt{OR} and \texttt{NOT}
	therefore looks as follows:

\begin{verbatim}
| 'NOT' expression                                # negateExpression
...
| expression op=AND expression                    # opExpression
| expression op=OR expression                     # opExpression
\end{verbatim}
	
	The \texttt{NOT} operator is transformed into an \texttt{UnaryExpression}
	whereas \texttt{AND} and \texttt{OR} become \texttt{BinaryExpression}s.

	For the \texttt{NOT} operator the evaluated value just needs to be XOR-ed with
	1. For the \texttt{AND} and \texttt{OR} operators the equivalent assembly
	instructions are being used (see \texttt{generateCode()}). Both operators
	require that the operands are \texttt{Boolean} values. This is verified in
	\texttt{refPass()}. The optimisations were implemented in \texttt{optimPass()}.

	\section{Pre-defined class Boolean}
	The type class \texttt{Boolean} is defined in \texttt{Types}. The following
	type checks were added to \texttt{ClassSymbol.isA()}:

\begin{verbatim}
if ((this eq Types.boolType) && (expected eq Types.boolClass)) {
  return true
}

if ((this eq Types.boolClass) && (expected eq Types.boolType)) {
  return true
}
\end{verbatim}

    Boxing was implemented in \texttt{Expression.generateCode()} by
    instantiating the \texttt{Boolean} class during runtime with the desired
    value as the first attribute.

	\section{Declaration of multiple user classes}
	To support multiple user classes only one minor change to the grammar was
	necessary:

\begin{verbatim}
program
  : classDeclaration*;
\end{verbatim}

	Then, for each class defined by a program, we add an entry to the list of
	user-defined classes in \texttt{Program} via \texttt{addClass()}. During the
	semantic analysis, this list is extended by all built-in classes that wrap primitive
	data types, namely \texttt{Object}, \texttt{Integer} and
	\texttt{Boolean}.

	\section{Methods with parameters}
	The changes to the grammar were trivial.

	When a new method is created, \texttt{MethodSymbol} must declare all parameters in the
	definition pass. Then, we assign an offset to each parameter
	denoting the position on the stack.

	When a method is called (\texttt{EvaluateExpression}), the arguments must
	be pushed on the stack, i.e. right before the return address.

	Also, the contextual analysis of \texttt{EvaluateExpression} takes into account
	any potential type mismatches. It also considers that the number of arguments
	may mismatch the number of method parameters.

	The object class \texttt{Types} contains all available built-in types and classes.
	Also, an attribute \texttt{\_value} is added to \texttt{intClass} und \texttt{boolClass}
	with the respective type.

	\section{Constant expressions}
	Scala allows matching class parameters. This feature facilitated the
	implementation of evaluating constant expressions during compile-time. The
	required transformations were implemented in \texttt{UnaryExpression} and
	\texttt{BinaryExpression}.

	\section{Methods with return values}
	A method \texttt{resolvedType()} was introduced in \texttt{Expression}. It
	is overridden in \texttt{EvaluateExpression} to return the result of a
	method (as opposed to \texttt{Void}).

\begin{verbatim}
override def resolvedType(): ClassSymbol =
  this.ref.declaration.get match {
    case v: VariableSymbol => v.getResolvedType
    case m: MethodSymbol => m.getResolvedReturnType
    case c: ClassSymbol => c
    case _ => super.resolvedType()
  }
\end{verbatim}

	Each \texttt{MethodSymbol} must therefore store the return type which defaults
	to \texttt{Void}. Minor sanity checks are performed during the referential
	phase such as the following:

\begin{verbatim}
if ((sem.currentClass.name == "Main") && (this.identifier.name == "main")) {
  if (this.parameters.size != 0) {
    throw new CompileException("Main.main() must not have any parameters.",
      this.identifier.position)
  } else if (this.getResolvedReturnType ne Types.voidType) {
    throw new CompileException("Main.main() must not have a non-void return type.",
      this.identifier.position)
  }
}
\end{verbatim}

	A \texttt{BranchEvaluator} was introduced to determine whether a method always
	terminates, i.e., whether at least one \texttt{RETURN} or \texttt{THROW} statement is reachable.
	This also takes into account loops and \texttt{IF} statements where the condition always
	evaluates to \texttt{true}. Knowing whether a method terminates with at least one branch
	allows further code optimisations; in \texttt{MethodSymbol} the method epilogue
	must only be generated if a method does not terminate and does not have a
	return value. This is due to the fact that the newly introduced
	\texttt{ReturnStatement} generates the method epilogue as well.

	The result of the \texttt{BranchEvaluator} is also used during the contextual
	analysis. An exception is thrown if a method is declared to have a return value
	even though it does not terminate (does not return anything or does not throw
	an exception).

	\section{Exception handling}
	At the very beginning of the assembly code of each program, the initial
	exception frame needs to be initialised. As the exception stack is rewinded
	traversing each \texttt{TRY} block, the default behaviour when no block is left is to jump
	to an internal function dealing with uncaught exceptions. This function is defined in the
	section \texttt{\_uncaughtException} which is situated right before
	\texttt{\_end} in order to terminate the program afterwards.

	The current exception frame is stored in
	the program segment \texttt{\_currentExceptionFrame} which is basically a linked
	list. An exception frame contains two elements. The first element denotes the
	exception handler (the label referencing the \texttt{TRY} block) and the second element
	the address pointing to the previous exception frame. When a new \texttt{TRY} block is
	entered, the 
	assembly code for pushing a new element on the exception stack is generated in \texttt{TryStatement}. In \texttt{ThrowStatement}
	we then store the exception code in a register and jump to the latest exception
	handler. In \texttt{TryStatement} this exception handler either catches the
	exception (if the exception code matches), or propagates the exception to the
	next exception hander. In both cases we need to pop one exception frame. When a
	normal \texttt{RETURN} takes place (\texttt{ReturnStatement}), we have to drop all
	exception frames (apart from the first one) as no exception occurred.

	This outlines the basic flow of throwing and catching an exception. Please see
	the particular classes for more detailed explanations.

	Error lists and \texttt{CATCH} alternatives are supported. In
	\texttt{Expression.generateDeRefCode()} we throw an exception when dereferencing
	instances pointing to \texttt{NULL}. Also, \texttt{BinaryExpression.generateCode()} throws an
	exception upon ``division by zero''.

	\section{Inheritance (virtual methods, BASE)}
	In each \texttt{ClassSymbol} the identifier of its super class must be stored.
	If there is none, the super class is \texttt{Object}. During the referential
	pass the super class is resolved to a \texttt{ClassSymbol} while checking
	for potential cycles.

	The referential pass is slightly more complicated than for the previous
	assignments; we need to verify whether all overridden methods have the same
	signature as its parent. Also, a method must not override an attribute in the
	super class and vice-versa. Finally, an index in the virtual method table (VMT)
	is assigned to each method, continuing the index of the super class. If a
	method is overridden, the VMT index from its parent method is inherited. In
	\texttt{isA()} the class hierarchy has to be taken into account, i.e., at least
	one class in the super class chain has to match.

	When calling a method in the super class, we need to take into account that a
	static call must be effectuated instead. The left operand is therefore \texttt{BASE}
	which is a variable declared in each class method. It
	has the same offset as the \texttt{SELF} variable. However, as the type of the super
	class is already known due to the inheritance chain during compile-time, it
	also has this type. In \texttt{AccessExpression} the static call is handled as
	follows:

\begin{verbatim}
this.leftOperand match {
  case call: EvaluateExpression =>
    if (call.ref.identifier.name == "BASE") {
      this.rightOperand.generateContextCode(true)
      this.rightOperand.setContext(
        call.ref.declaration.get.asInstanceOf[VariableSymbol], true)
    }
  case _ =>
}
\end{verbatim}

    For each registered class, a VMT is generated in the assembly code. This
    is performed in \texttt{Program}. The first entry of the VMT is the reference
    to the VMT of the super class. The latter is needed for dynamic type casts.

    When a class is instantiated using \texttt{NewStatement}, the address to
    VMT of the respective class is written at position 0 of the resulting
    object.

	A dynamic call is effectuated by obtaining the VMT from the object. Then, the
	function address can be resolved from the VMT as the method offset is known
	beforehand.

	The logic for validating the access levels was mainly implemented in
	\texttt{Scope.resolve()} under the use of the helper function
	\texttt{Symbol.availableFor()}. The levels of overridden methods are validated
	in the referential pass of \texttt{MethodSymbol}.

	\section{Type checking and type casts}
	The \texttt{TypeCheckExpression} implements the logic for dynamic type
	checking. This is implemented by first resolving the object's VMT and then
	following the class hierarchy until a match was found. To prevent replication,
	a helper function \texttt{checkType()} was written. It is also used in
	\texttt{EvaluateExpression} to check whether a conversion is valid. If so, the
	top of the stack contains already the correct value. Otherwise, the value
	is replaced by \texttt{NULL}.

	\section{Garbage collection}
	Due to time constraints it was not possible to implement garbage collection.

%\bibliographystyle{abbrv}
\bibliography{impl}

\end{document}